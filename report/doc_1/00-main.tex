\documentclass[11pt,oneside,a4paper,article,brazil,chapter=title]{abntex2}
\usepackage[top=2cm,bottom=2cm,left=3cm,right=3cm,marginparwidth=1.75cm]{geometry}
\usepackage[utf8]{inputenc}
\usepackage{times}
\usepackage{graphicx} % Required for inserting images
\usepackage{svg}
\usepackage{float}
\usepackage{csquotes}
\usepackage{amsmath}
\usepackage{amssymb}
\usepackage{amsfonts}
\usepackage{amsthm}
\usepackage{indentfirst}
\usepackage{url}
\usepackage[alf,num]{abntex2cite}
\usepackage{pdflscape}
\usepackage{caption}
\usepackage{hyperref}

\hypersetup{
  colorlinks=true,
  linkcolor=black,
  citecolor=magenta,
  urlcolor=magenta
}

\renewcommand\brazilhyphenmins{22}

\titulo{
  Primeiro Trabalho Prático\\
  \vspace{8pt}
  As Cinzas do Brasil \\
  \small{Análise dos dados brasileiros de queimadas}
}
\author{
  Clara Ernesto de Carvalho \; \textit{14559479}\\
  Felipe Carneiro Machado \; \textit{14569373}\\
  Lívia Lelis \; \textit{12543822}
}
\data{10/2025}
\instituicao{Universidade de São Paulo - USP}
\newcommand{\departamento}{Departamento de Computação}
\newcommand{\edc}{Instituto De Ciências Matemáticas e de Computação - ICMC}
\local{São Carlos - SP}
\newcommand{\tipo}{SCC0245 - Processamento Analítico de Dados}

\renewcommand{\imprimircapa}{
  \begin{center}

    {\ABNTEXchapterfont\LARGE\MakeUppercase\imprimirinstituicao} \par
    {\ABNTEXchapterfont\large\uppercase{\edc}}

    {\noindent\rule{\textwidth}{0.1pt}}

    {\ABNTEXchapterfont\normalsize\tipo}
    \begin{flushright}
      Profa. Cristina Dutra de Aguiar
    \end{flushright}
    \vspace*{\fill}

    \vspace*{\fill}
    {\ABNTEXchapterfont\bfseries\LARGE\imprimirtitulo}
    \vspace*{\fill}

    \begin{flushright}
      Clara Ernesto de Carvalho \; \textit{14559479}\\
      Felipe Carneiro Machado \; \textit{14569373}\\
      Lívia Lelis \; \textit{12543822}
    \end{flushright}

    \vspace*{\fill}
    {\large\imprimirlocal} \par
    {\large\imprimirdata}
    \vspace*{1cm}

  \end{center}
}

\begin{document}

\selectlanguage{brazil}

\imprimircapa
\clearpage

\pdfbookmark[0]{\contentsname}{toc}

\tableofcontents*
\cleardoublepage

\textual

\section{Descrição do Problema}

O Brasil sofre com o volumoso problema que são as queimadas, segundo dados do Monitor do Fogo do projeto MapBiomas, mais de 30,8 milhões de hectares foram queimados no Brasil entre janeiro e dezembro de 2024, uma área maior que todo o território da Itália. \cite{mapbiomas_2025_area_queimada}

Nesse contexto, o nosso foco é analisar os dados de queimada e clima informados pelo INPE, com o objetivo de encontrar correlações entre os dados, entender como eles se comportam em relação à localização geográfica, época do ano, e como se alteraram ao longo dos anos; e criar um data warehouse para facilitar essas consultas uma vez que possuímos um grande volume de dados.

Nós usaremos dois datasets, o dataset do INPE de queimadas \cite{queimadas_inpe} e o de clima \cite{sisam_inpe}. Ambos possuem dados com localização espacial e temporal, o de clima possui dados desde 2000, e o de queimadas desde 2003. Devido à longa coleta de dados, ambos datasets são muito volumosos, sendo que o dataset de queimadas possui 17.542.900 tuplas, e o dataset de clima possui 158.705.816, totalizando em mais de 15GB. Além disso, precisaremos dos diretórios brasileiros de unidades federativas e municípios.

\subsection{Fontes dos Dados}

\subsubsection{Queimadas - INPE}

A Tabela \ref{tab:queimadas_schema} apresenta o esquema do dataset de queimadas fornecido pelo INPE, contendo informações sobre focos de incêndio detectados por satélite.

\begin{table}[H]
\centering
\caption{Esquema do Dataset de Queimadas - INPE}
\label{tab:queimadas_schema}
\resizebox{\textwidth}{!}{
\begin{tabular}{|l|l|l|p{8cm}|}
\hline
\textbf{Atributo} & \textbf{Tipo} & \textbf{Modo} & \textbf{Descrição} \\ \hline
ano & INTEGER & NULLABLE & Ano de referência da passagem do satélite segundo o fuso horário de Greenwich (GMT). \\ \hline
mes & INTEGER & NULLABLE & Mês de referência da passagem do satélite segundo o fuso horário de Greenwich (GMT). \\ \hline
data\_hora & DATETIME & NULLABLE & Data e hora da passagem do satélite no fuso horário de Greenwich (GMT). \\ \hline
bioma & STRING & NULLABLE & Nome do Bioma. \\ \hline
sigla\_uf & STRING & NULLABLE & Sigla da Unidade Federativa. \\ \hline
id\_municipio & STRING & NULLABLE & Identificação do Município. \\ \hline
latitude & FLOAT & NULLABLE & Latitude do centro do píxel de fogo ativo em graus decimais. \\ \hline
longitude & FLOAT & NULLABLE & Longitude do centro do píxel de fogo ativo em graus decimais. \\ \hline
satelite & STRING & NULLABLE & Nome do algoritmo utilizado e referência ao satélite provedor da imagem. \\ \hline
dias\_sem\_chuva & FLOAT & NULLABLE & Número de dias sem chuva até a detecção do foco. \\ \hline
precipitacao & FLOAT & NULLABLE & Precipitação acumulada no dia até o momento da detecção do fogo. \\ \hline
risco\_fogo & FLOAT & NULLABLE & Valor do Risco de Fogo previsto para o dia da detecção do foco. \\ \hline
potencia\_radiativa\_fogo & FLOAT & NULLABLE & Fire Radiative Power, MW (megawatts). \\ \hline
\end{tabular}
}
\end{table}

\subsubsection{Clima - SISAM/INPE}

A Tabela \ref{tab:sisam_schema} apresenta o esquema do dataset de clima fornecido pelo Sistema de Informações Ambientais Integrado à Saúde Ambiental (SISAM) do INPE, contendo dados meteorológicos e de qualidade do ar.

\begin{table}[H]
\centering
\caption{Esquema do Dataset de Clima - SISAM/INPE}
\label{tab:sisam_schema}
\resizebox{\textwidth}{!}{
\begin{tabular}{|l|l|l|p{8cm}|}
\hline
\textbf{Atributo} & \textbf{Tipo} & \textbf{Modo} & \textbf{Descrição} \\ \hline
ano & INTEGER & NULLABLE & Ano do registro. \\ \hline
sigla\_uf & STRING & NULLABLE & Sigla da Unidade da Federação. \\ \hline
id\_municipio & STRING & NULLABLE & ID do Município segundo classificação IBGE (7 dígitos). \\ \hline
data\_hora & DATETIME & NULLABLE & Data e hora do registro meteorológico. \\ \hline
co\_ppb & FLOAT & NULLABLE & Concentração de monóxido de carbono (CO) em partes por bilhão (ppb). \\ \hline
no2\_ppb & FLOAT & NULLABLE & Concentração de dióxido de nitrogênio (NO$_2$) em partes por bilhão (ppb). \\ \hline
o3\_ppb & FLOAT & NULLABLE & Concentração de ozônio (O$_3$) em partes por bilhão (ppb). \\ \hline
pm25\_ugm3 & FLOAT & NULLABLE & Material particulado fino (PM2.5) em microgramas por metro cúbico ($\mu$g/m³). \\ \hline
so2\_ugm3 & FLOAT & NULLABLE & Dióxido de enxofre (SO$_2$) em microgramas por metro cúbico ($\mu$g/m³). \\ \hline
precipitacao\_dia & FLOAT & NULLABLE & Total de precipitação acumulada no momento do registro. \\ \hline
temperatura & FLOAT & NULLABLE & Temperatura do ar (°C). \\ \hline
umidade\_relativa & FLOAT & NULLABLE & Umidade relativa do ar (\%). \\ \hline
vento\_direcao & INTEGER & NULLABLE & Direção do vento (graus azimutais). \\ \hline
vento\_velocidade & FLOAT & NULLABLE & Velocidade do vento (m/s). \\ \hline
\end{tabular}
}
\end{table}

\subsubsection{Unidades Federativas - IBGE}

A Tabela \ref{tab:uf_schema} apresenta o esquema do dataset de Unidades Federativas fornecido pelo IBGE, contendo informações sobre os estados brasileiros e suas regiões.

\begin{table}[H]
\centering
\caption{Esquema do Dataset de Unidades Federativas - IBGE}
\label{tab:uf_schema}
\begin{tabular}{|l|l|l|p{8cm}|}
\hline
\textbf{Atributo} & \textbf{Tipo} & \textbf{Modo} & \textbf{Descrição} \\ \hline
id\_uf & INTEGER & NOT NULL & Código identificador da Unidade da Federação (2 dígitos). \\ \hline
sigla & STRING & NOT NULL & Sigla da Unidade da Federação. \\ \hline
nome & STRING & NOT NULL & Nome completo da Unidade da Federação. \\ \hline
regiao & STRING & NOT NULL & Nome da região geográfica (Norte, Nordeste, Centro-Oeste, Sudeste, Sul). \\ \hline
\end{tabular}
\end{table}

\subsubsection{Municípios - IBGE}

A Tabela \ref{tab:municipio_schema} apresenta o esquema do dataset de Municípios fornecido pelo IBGE, contendo informações detalhadas sobre todos os municípios brasileiros, incluindo divisões administrativas e geográficas.

\begin{table}[H]
\centering
\caption{Esquema do Dataset de Municípios - IBGE}
\label{tab:municipio_schema}
\resizebox{\textwidth}{!}{
\begin{tabular}{|l|l|l|p{7cm}|}
\hline
\textbf{Atributo} & \textbf{Tipo} & \textbf{Modo} & \textbf{Descrição} \\ \hline
id\_municipio & INTEGER & NOT NULL & Código identificador do município IBGE (7 dígitos). \\ \hline
id\_municipio\_6 & INTEGER & NULLABLE & Código do município com 6 dígitos. \\ \hline
id\_municipio\_tse & INTEGER & NULLABLE & Código do município no Tribunal Superior Eleitoral. \\ \hline
id\_municipio\_rf & INTEGER & NULLABLE & Código do município na Receita Federal. \\ \hline
id\_municipio\_bcb & INTEGER & NULLABLE & Código do município no Banco Central do Brasil. \\ \hline
nome & STRING & NOT NULL & Nome do município. \\ \hline
capital\_uf & INTEGER & NULLABLE & Indicador se o município é capital (1) ou não (0). \\ \hline
id\_comarca & INTEGER & NULLABLE & Código identificador da comarca. \\ \hline
id\_regiao\_saude & INTEGER & NULLABLE & Código da região de saúde. \\ \hline
nome\_regiao\_saude & STRING & NULLABLE & Nome da região de saúde. \\ \hline
id\_regiao\_imediata & INTEGER & NULLABLE & Código da região geográfica imediata. \\ \hline
nome\_regiao\_imediata & STRING & NULLABLE & Nome da região geográfica imediata. \\ \hline
id\_regiao\_intermediaria & INTEGER & NULLABLE & Código da região geográfica intermediária. \\ \hline
nome\_regiao\_intermediaria & STRING & NULLABLE & Nome da região geográfica intermediária. \\ \hline
id\_microrregiao & INTEGER & NULLABLE & Código da microrregião. \\ \hline
nome\_microrregiao & STRING & NULLABLE & Nome da microrregião. \\ \hline
id\_mesorregiao & INTEGER & NULLABLE & Código da mesorregião. \\ \hline
nome\_mesorregiao & STRING & NULLABLE & Nome da mesorregião. \\ \hline
id\_regiao\_metropolitana & STRING & NULLABLE & Lista de IDs de regiões metropolitanas (separados por vírgula). \\ \hline
nome\_regiao\_metropolitana & STRING & NULLABLE & Lista de nomes de regiões metropolitanas (separados por vírgula). \\ \hline
ddd & INTEGER & NULLABLE & Código DDD de telefonia. \\ \hline
id\_uf & INTEGER & NOT NULL & Código da Unidade da Federação. \\ \hline
sigla\_uf & STRING & NOT NULL & Sigla da Unidade da Federação. \\ \hline
nome\_uf & STRING & NOT NULL & Nome da Unidade da Federação. \\ \hline
nome\_regiao & STRING & NOT NULL & Nome da região geográfica. \\ \hline
amazonia\_legal & INTEGER & NULLABLE & Indicador se pertence à Amazônia Legal (1) ou não (0). \\ \hline
centroide & STRING & NULLABLE & Coordenadas geográficas do centroide do município. \\ \hline
\end{tabular}
}
\end{table}

\subsection{Assuntos de Interesse}

\subsubsection{FRP - Fire Radiative Power}

Impacto da Queimada - Indica (em MW) a potência radiativa da queimada, relacionada com o quanto de material foi consumido, e portanto, o quanto que a queimada impactou. Ela pode ser analisada em relação aos locais em que ocorre, e ao tempo. A métrica (fato) observada é o FRP e a granularidade desta métrica está por minuto e local. As dimensões relacionadas são data, horário e local.

\subsubsection{Dias sem Chuva}

Analisar o impacto dos períodos de seca nas queimadas, e em quais locais ocorrem mais queimadas mesmo fora destes períodos. Além disso, correlacionar a duração destes períodos com o impacto da queimada (FRP). A métrica (fato) observada é a dias\_sem\_chuva e a granularidade desta métrica está por minuto e local. As dimensões relacionadas são data e local.

\subsubsection{Umidade Relativa}

Analisar umidade em relação ao local, data e horário para encontrar locais e momentos mais afetados por baixa umidade e correlacionar com o impacto da queimada (FRP). A métrica (fato) observada é a umidade\_relativa e a granularidade desta métrica está por hora e local. As dimensões relacionadas são data, horário e local.

\subsubsection{Risco de Fogo}

Analisar a previsão do risco de queimada em relação a local e data, e analisar se há correlação com o impacto das queimadas. (Se a previsão da ocorrência de queimadas se correlaciona com o impacto das queimadas existentes no mesmo período).

A métrica (fato) observada é a risco\_fogo e a granularidade desta métrica está por minuto e local. As dimensões relacionadas são data e local.

\subsection{Dimensões e Fatos}

Para a modelagem do Data Warehouse, os dados serão organizados em uma \textbf{constelação de fatos}, composta por duas tabelas de fatos principais (Queimadas e Clima) e um conjunto de \textbf{dimensões conformadas}. As dimensões conformadas são compartilhadas entre as tabelas de fato, garantindo consistência e permitindo análises integradas (\textit{drill-across}).

\subsubsection{Dimensões Conformadas}
As dimensões que darão contexto aos fatos são:
\begin{itemize}
    \item \textbf{Dimensão de Tempo (Data e Horário)}: Para analisar os eventos ao longo do tempo, com granularidades que vão de minuto a ano. Uma vez que os fatos de Queimadas e Clima possuem granularidades temporais distintas (minuto e hora, respectivamente), será utilizada uma tabela ponte para compatibilizá-las.
    \item \textbf{Dimensão de Localização}: Para análises geográficas, com hierarquias que vão desde coordenadas/municípios até regiões e biomas.
\end{itemize}

\subsubsection{Tabelas de Fato}
As tabelas de fato conterão as métricas quantitativas dos eventos. A correta classificação da aditividade de cada fato é crucial para a análise.

\paragraph{Fato Queimadas:}
\begin{itemize}
    \item \textbf{Potência Radiativa do Fogo (FRP)}: É um fato \textbf{aditivo}. A energia liberada pode ser somada através de qualquer dimensão (e.g., total de FRP para um estado em um mês).
    \item \textbf{Dias sem Chuva}: É um fato \textbf{semi-aditivo}. Pode ser agregado por média entre locais, mas não é somável ao longo do tempo. Para análises temporais, utilizam-se funções como `MAX`, `MIN` ou a última medição.
    \item \textbf{Risco de Fogo}: É um fato \textbf{não aditivo}. Não pode ser somado de forma útil. A agregação correta é a média para entender o risco em uma determinada região ou período.
\end{itemize}

\paragraph{Fato Clima:}
\begin{itemize}
    \item \textbf{Temperatura, Umidade Relativa, Velocidade do Vento}: São fatos \textbf{não aditivos}. A agregação deve ser feita por média.
    \item \textbf{Precipitação}: É um fato \textbf{semi-aditivo}. A precipitação acumulada pode ser somada entre diferentes locais para um mesmo período, mas não ao longo do tempo.
\end{itemize}

\subsubsection{Processo de Integração}

A estratégia de integração adotada consiste em:

\begin{enumerate}
    \item \textbf{Integração Direta por Chaves Naturais}: Os campos \texttt{id\_municipio} e \texttt{sigla\_uf} servem como chaves naturais para unificar os dados geográficos entre as fontes.
    
    \item \textbf{Enriquecimento com Dados IBGE}: Os diretórios de municípios e unidades federativas do IBGE são utilizados para enriquecer as dimensões conformadas com informações hierárquicas (região, mesorregião, microrregião).
    
    \item \textbf{Dimensões Conformadas}: A criação de dimensões conformadas (\texttt{dim\_data}, \texttt{dim\_local}, \texttt{dim\_horario}) garante consistência semântica entre os fatos de queimadas e clima.
    
    \item \textbf{Resolução de Granularidade Temporal}: Uma tabela ponte (\texttt{bridge\_horario}) resolve a diferença de granularidade entre minutos (queimadas) e horas (clima), permitindo análises drill-across.
\end{enumerate}

\subsubsection{Ausência de Conflitos Semânticos}

Como todos os datasets originam-se de fontes oficiais brasileiras (INPE e IBGE) e seguem padrões governamentais de codificação, \textbf{não há necessidade de reconciliação semântica} ou resolução de conflitos. A integração é essencialmente uma operação de junção por chaves naturais, seguida de desnormalização controlada para o modelo dimensional.

\subsection{Mapeamento de Atributos: Fonte para Data Warehouse}

Esta seção documenta o mapeamento completo entre os atributos das fontes de dados e os elementos do Data Warehouse (dimensões e fatos).

\subsubsection{Dimensão: dim\_data}

\begin{table}[H]
\centering
\caption{Mapeamento de Atributos - dim\_data}
\label{tab:mapping_dim_data}
\resizebox{\textwidth}{!}{
\begin{tabular}{|l|l|l|p{6cm}|}
\hline
\textbf{Atributo DW} & \textbf{Fonte} & \textbf{Atributo Fonte} & \textbf{Transformação} \\ \hline
id & - & - & Chave surrogate gerada \\ \hline
ano & Queimadas/Clima & ano & Extração direta \\ \hline
semestre & Queimadas/Clima & data\_hora & Calculado: 1 (jan-jun) ou 2 (jul-dez) \\ \hline
trimestre & Queimadas/Clima & data\_hora & Calculado: \texttt{ceil(mes/3)} \\ \hline
mês & Queimadas/Clima & mes & Extração direta \\ \hline
dia & Queimadas/Clima & data\_hora & Extração do dia do campo datetime \\ \hline
dia\_da\_semana & Queimadas/Clima & data\_hora & Calculado (1=domingo, 7=sábado) \\ \hline
dia\_do\_ano & Queimadas/Clima & data\_hora & Calculado (1-366) \\ \hline
numero\_semana & Queimadas/Clima & data\_hora & Calculado (ISO week number) \\ \hline
fim\_de\_semana & Queimadas/Clima & data\_hora & Calculado: \texttt{dia\_da\_semana IN (1,7)} \\ \hline
estacao & Queimadas/Clima & data\_hora & Mapeamento por mês (hemisfério sul) \\ \hline
\end{tabular}
}
\end{table}

\subsubsection{Dimensão: dim\_local}

\begin{table}[H]
\centering
\caption{Mapeamento de Atributos - dim\_local}
\label{tab:mapping_dim_local}
\resizebox{\textwidth}{!}{
\begin{tabular}{|l|l|l|p{6cm}|}
\hline
\textbf{Atributo DW} & \textbf{Fonte} & \textbf{Atributo Fonte} & \textbf{Transformação} \\ \hline
id & - & - & Chave surrogate gerada \\ \hline
municipio & Municípios IBGE & nome & Extração direta \\ \hline
estado & Municípios IBGE & nome\_uf & Extração direta \\ \hline
região & Municípios IBGE & nome\_regiao & Extração direta \\ \hline
\multicolumn{4}{|l|}{\textit{Chave Natural de Integração: id\_municipio (7 dígitos IBGE)}} \\ \hline
\end{tabular}
}
\end{table}

\subsubsection{Dimensão: dim\_ponto}

\begin{table}[H]
\centering
\caption{Mapeamento de Atributos - dim\_ponto}
\label{tab:mapping_dim_ponto}
\resizebox{\textwidth}{!}{
\begin{tabular}{|l|l|l|p{6cm}|}
\hline
\textbf{Atributo DW} & \textbf{Fonte} & \textbf{Atributo Fonte} & \textbf{Transformação} \\ \hline
id & - & - & Chave surrogate gerada \\ \hline
local\_fk & dim\_local & id & Lookup via id\_municipio \\ \hline
latitude & Queimadas & latitude & Extração direta (graus decimais) \\ \hline
longitude & Queimadas & longitude & Extração direta (graus decimais) \\ \hline
bioma & Queimadas & bioma & Extração direta \\ \hline
\end{tabular}
}
\end{table}

\subsubsection{Dimensão: dim\_horario}

\begin{table}[H]
\centering
\caption{Mapeamento de Atributos - dim\_horario}
\label{tab:mapping_dim_horario}
\resizebox{\textwidth}{!}{
\begin{tabular}{|l|l|l|p{6cm}|}
\hline
\textbf{Atributo DW} & \textbf{Fonte} & \textbf{Atributo Fonte} & \textbf{Transformação} \\ \hline
id & - & - & Chave surrogate gerada \\ \hline
hora & Queimadas/Clima & data\_hora & Extração da hora do campo datetime \\ \hline
minuto & Queimadas/Clima & data\_hora & Extração do minuto do campo datetime \\ \hline
\end{tabular}
}
\end{table}

\subsubsection{Tabela Ponte: bridge\_horario}

\begin{table}[H]
\centering
\caption{Mapeamento de Atributos - bridge\_horario}
\label{tab:mapping_bridge_horario}
\resizebox{\textwidth}{!}{
\begin{tabular}{|l|l|l|p{6cm}|}
\hline
\textbf{Atributo DW} & \textbf{Fonte} & \textbf{Atributo Fonte} & \textbf{Transformação} \\ \hline
horario\_minuto\_fk & dim\_horario & id & Lookup para granularidade de minuto \\ \hline
horario\_hora\_fk & dim\_horario & id & Lookup para granularidade de hora (minuto=0) \\ \hline
\multicolumn{4}{|l|}{\textit{Relação: cada hora (HH:00) mapeia para 60 minutos (HH:00 a HH:59)}} \\ \hline
\end{tabular}
}
\end{table}

\subsubsection{Fato: Queimadas}

\begin{table}[H]
\centering
\caption{Mapeamento de Atributos - Fato Queimadas}
\label{tab:mapping_fact_queimadas}
\resizebox{\textwidth}{!}{
\begin{tabular}{|l|l|l|l|p{5cm}|}
\hline
\textbf{Atributo DW} & \textbf{Tipo} & \textbf{Fonte} & \textbf{Atributo Fonte} & \textbf{Transformação} \\ \hline
\multicolumn{5}{|c|}{\textit{Chaves Estrangeiras (Dimensões)}} \\ \hline
data\_fk & FK & Queimadas & data\_hora, ano, mes & Lookup em dim\_data \\ \hline
ponto\_fk & FK & Queimadas & lat, long, bioma, id\_municipio & Lookup em dim\_ponto \\ \hline
horario\_fk & FK & Queimadas & data\_hora & Lookup em dim\_horario (minuto) \\ \hline
\multicolumn{5}{|c|}{\textit{Fatos (Métricas)}} \\ \hline
risco\_fogo & Fato & Queimadas & risco\_fogo & Extração direta (não aditivo) \\ \hline
potencia\_radiativa\_fogo & Fato & Queimadas & potencia\_radiativa\_fogo & Extração direta (aditivo, MW) \\ \hline
dias\_sem\_chuva & Fato & Queimadas & dias\_sem\_chuva & Extração direta (semi-aditivo) \\ \hline
\multicolumn{5}{|l|}{\textit{Granularidade: um foco de incêndio por ponto geográfico, data e minuto}} \\ \hline
\end{tabular}
}
\end{table}

\subsubsection{Fato: Clima}

\begin{table}[H]
\centering
\caption{Mapeamento de Atributos - Fato Clima}
\label{tab:mapping_fact_clima}
\resizebox{\textwidth}{!}{
\begin{tabular}{|l|l|l|l|p{5cm}|}
\hline
\textbf{Atributo DW} & \textbf{Tipo} & \textbf{Fonte} & \textbf{Atributo Fonte} & \textbf{Transformação} \\ \hline
\multicolumn{5}{|c|}{\textit{Chaves Estrangeiras (Dimensões)}} \\ \hline
data\_fk & FK & Clima & data\_hora, ano & Lookup em dim\_data \\ \hline
local\_fk & FK & Clima & id\_municipio, sigla\_uf & Lookup em dim\_local \\ \hline
horario\_fk & FK & Clima & data\_hora & Lookup em dim\_horario (hora) \\ \hline
\multicolumn{5}{|c|}{\textit{Fatos (Métricas)}} \\ \hline
temperatura & Fato & Clima & temperatura & Extração direta (°C, não aditivo) \\ \hline
precipitacao & Fato & Clima & precipitacao\_dia & Extração direta (mm, semi-aditivo) \\ \hline
umidade\_relativa & Fato & Clima & umidade\_relativa & Extração direta (\%, não aditivo) \\ \hline
vento\_velocidade & Fato & Clima & vento\_velocidade & Extração direta (m/s, não aditivo) \\ \hline
vento\_direcao & Fato & Clima & vento\_direcao & Extração direta (graus, não aditivo) \\ \hline
co\_ppb & Fato & Clima & co\_ppb & Extração direta (ppb, não aditivo) \\ \hline
no2\_ppb & Fato & Clima & no2\_ppb & Extração direta (ppb, não aditivo) \\ \hline
o3\_ppb & Fato & Clima & o3\_ppb & Extração direta (ppb, não aditivo) \\ \hline
pm25\_ugm3 & Fato & Clima & pm25\_ugm3 & Extração direta ($\mu$g/m³, não aditivo) \\ \hline
so2\_ugm3 & Fato & Clima & so2\_ugm3 & Extração direta ($\mu$g/m³, não aditivo) \\ \hline
\multicolumn{5}{|l|}{\textit{Granularidade: medições climáticas agregadas por município, data e hora}} \\ \hline
\end{tabular}
}
\end{table}



\section{Esquemas}

Os fatos e dimensões foram organizados em uma constelação de fatos composta por 2 esquemas estrela: Queimadas, responsável por armazenar dados relativos à ocorrência de queimadas, e Clima, que armazena dados climáticos históricos. A integração entre os esquemas é realizada através de \textbf{dimensões conformadas} (conformed dimensions), que são compartilhadas entre os fatos, permitindo análises integradas e consistentes.

\subsection{Dimensões Conformadas}

Para integrar os esquemas Queimadas e Clima de forma consistente, foram criadas \textbf{dimensões conformadas} que são compartilhadas entre ambos os fatos. Esta abordagem permite análises cruzadas e drill-across entre os dois domínios de dados.

\subsubsection{dim\_data}
Dimensão temporal compartilhada por ambos os fatos:
\begin{itemize}
    \item \textbf{id}: long, chave primária (PK).
    \item \textbf{ano}: int.
    \item \textbf{semestre}: int.
    \item \textbf{trimestre}: int.
    \item \textbf{mês}: int.
    \item \textbf{dia}: int.
    \item \textbf{dia\_da\_semana}: int (1-7, onde 1=domingo).
    \item \textbf{dia\_do\_ano}: int (1-366).
    \item \textbf{numero\_semana}: int (1-53).
    \item \textbf{fim\_de\_semana}: boolean.
    \item \textbf{estacao}: string (Verão, Outono, Inverno, Primavera).
    \item \textbf{Hierarquia}: dia \(\rightarrow\) mês \(\rightarrow\) trimestre \(\rightarrow\) semestre \(\rightarrow\) ano
\end{itemize}

\paragraph{Atributos Analíticos Adicionais}
Os atributos \textit{dia\_da\_semana}, \textit{fim\_de\_semana}, \textit{numero\_semana}, \textit{dia\_do\_ano} e \textit{estacao} foram incluídos para facilitar análises específicas:
\begin{itemize}
    \item \textbf{Análises Sazonais}: O atributo \textit{estacao} permite estudos sobre padrões de queimadas e condições climáticas por estações do ano, fundamental para entender ciclos naturais de seca e chuvas.
    \item \textbf{Análises Semanais}: \textit{dia\_da\_semana} e \textit{fim\_de\_semana} facilitam a identificação de padrões relacionados a atividades humanas, como queimadas controladas durante dias úteis.
    \item \textbf{Análises de Períodos Específicos}: \textit{numero\_semana} e \textit{dia\_do\_ano} permitem comparações diretas entre períodos equivalentes de anos diferentes, essenciais para análises de tendências temporais.
\end{itemize}

\subsubsection{dim\_local}
Dimensão geográfica conformada, no grão de \textbf{município}.
\begin{itemize}
    \item \textbf{id}: long, chave primária (PK).
    \item \textbf{municipio}: string.
    \item \textbf{estado}: string.
    \item \textbf{região}: string.
    \item \textbf{Hierarquia}: município \(\rightarrow\) estado \(\rightarrow\) região.
\end{itemize}

\subsubsection{dim\_ponto}
Dimensão essencial para análise de queimadas por biomas e regiões geográficas específicas. Esta dimensão mantém o grão fino necessário para análises ecológicas e de conservação, permitindo estudos sobre padrões de queimadas em diferentes ecossistemas brasileiros.
\begin{itemize}
    \item \textbf{id}: long, chave primária (PK).
    \item \textbf{local\_fk}: long, FK para \textit{dim\_local}.
    \item \textbf{latitude}: float.
    \item \textbf{longitude}: float.
    \item \textbf{bioma}: string. Fundamental para análises por Amazônia, Cerrado, Mata Atlântica, etc.
\end{itemize}

Esta estrutura hierárquica (ponto \(\rightarrow\) município \(\rightarrow\) estado \(\rightarrow\) região) é fundamental para o domínio de queimadas, pois permite análises em múltiplas granularidades geográficas e por biomas. Enquanto o fato de clima opera no grão de município, o fato de queimadas requer granularidade mais fina para capturar a diversidade ecológica e espacial dos incêndios. Análises \textit{drill-across} são realizadas através do relacionamento hierárquico entre \textit{dim\_ponto} e \textit{dim\_local}.

\subsection{Slowly Changing Dimensions (SCDs)}

Embora as dimensões geográficas (\textit{dim\_local} e \textit{dim\_ponto}) possam, em teoria, sofrer alterações ao longo do tempo (como mudanças nos nomes de municípios, criação de novos municípios, ou alterações em limites geográficos), optamos por \textbf{não implementar SCDs} para essas dimensões neste projeto pelas seguintes razões:

\subsubsection{Justificativa para Não Implementação de SCDs}
\begin{itemize}
    \item \textbf{Escopo do Projeto}: O foco principal está na análise de padrões de queimadas e correlações climáticas, onde mudanças geográficas administrativas não impactam significativamente os resultados analíticos.
    
    \item \textbf{Natureza dos Dados}: Os dados de queimadas são eventos pontuais no tempo e espaço. O que importa é a localização exata onde o evento ocorreu no momento específico, não mudanças administrativas posteriores.
    
    \item \textbf{Complexidade vs. Benefício}: A implementação de SCDs adicionaria complexidade significativa ao modelo sem fornecer valor analítico proporcional para os objetivos deste projeto.
    
    \item \textbf{Integridade Temporal dos Fatos}: Os fatos de queimadas sempre referenciam as características geográficas apropriadas ao momento em que o evento ocorreu, garantindo consistência temporal e precisão analítica.
\end{itemize}

\subsubsection{Abordagem Adotada}
As dimensões geográficas mantêm seus valores conforme registrados no momento da ocorrência dos eventos. Esta abordagem:
\begin{itemize}
    \item Preserva a integridade histórica dos dados.
    \item Mantém a simplicidade do modelo para consultas e análises.
    \item Assegura que cada fato de queimada seja sempre analisado no contexto geográfico correto de quando ocorreu.
    \item Facilita análises longitudinais sem a complexidade de versionamento dimensional.
\end{itemize}

\subsubsection{dim\_horario}
Dimensão de horário compartilhada, com granularidade mínima de minuto:
\begin{itemize}
    \item \textbf{id}: long, chave primária (PK).
    \item \textbf{hora}: int.
    \item \textbf{minuto}: int.
    \item \textbf{Hierarquia}: minuto \(\rightarrow\) hora
\end{itemize}

\subsection{Tabela Ponte (Bridge Table) para Tempo}

Devido à diferença de granularidade temporal entre os fatos Queimadas (minuto) e Clima (hora), foi criada uma \textbf{tabela ponte} (\textit{bridge\_horario}) para facilitar consultas que relacionem os dois fatos no domínio temporal. A granularidade de minuto é essencial para queimadas, pois múltiplos focos de incêndio podem ocorrer no mesmo município dentro da mesma hora, e agregá-los resultaria em perda significativa de informações analíticas sobre a intensidade e distribuição temporal dos eventos.

\subsubsection{bridge\_horario}
\begin{itemize}
    \item \textbf{horario\_minuto\_fk}: long, FK para dim\_horario (granularidade minuto).
    \item \textbf{horario\_hora\_fk}: long, FK para dim\_horario (granularidade hora).
\end{itemize}

\subsubsection{Mecânica da Tabela Ponte}
Esta tabela implementa um relacionamento \textbf{muitos-para-um (60:1)}, onde cada registro de hora em \textit{dim\_horario} se relaciona com exatamente 60 registros de minutos correspondentes. Por exemplo:
\begin{itemize}
    \item Hora 14:00 $\rightarrow$ Minutos 14:00, 14:01, 14:02, ..., 14:59
    \item Hora 15:00 $\rightarrow$ Minutos 15:00, 15:01, 15:02, ..., 15:59
\end{itemize}

Esta estrutura permite três tipos de consultas temporais:
\begin{enumerate}
    \item \textbf{Análise granular de queimadas}: Consultas diretas em granularidade de minuto
    \item \textbf{Análise climática}: Consultas em granularidade de hora
    \item \textbf{Análise integrada (drill-across)}: Agregação de queimadas por hora para comparação com dados climáticos
\end{enumerate}

\subsubsection{Alternativas Arquiteturais Consideradas}
\paragraph{Abordagem 1: Fato Único com Granularidade Mínima}
Criar um único fato com granularidade de minuto forçaria a interpolação artificial de dados climáticos horários, introduzindo imprecisão e aumentando desnecessariamente o volume de dados em 60 vezes.

\paragraph{Abordagem 2: Fato Único com Granularidade Máxima}
Utilizar apenas granularidade de hora resultaria na perda de informações críticas sobre a distribuição temporal de queimadas, essencial para análises de intensidade e padrões de propagação.

\paragraph{Abordagem 3: Dimensões Temporais Separadas}
Manter dimensões temporais completamente separadas impediria consultas drill-across, eliminando a capacidade de correlacionar eventos climáticos e de queimadas.

\paragraph{Justificativa da Solução Adotada}
A tabela ponte oferece \textbf{flexibilidade máxima} mantendo a integridade dos dados originais. Permite agregações controladas quando necessário, preserva a granularidade natural de cada domínio, e facilita tanto análises especializadas quanto integradas.

\subsection{Fato Queimadas}

\subsubsection{Granularidade}
A tabela de fatos Queimadas opera no grão de \textbf{um foco de incêndio detectado por satélite em um ponto geográfico específico (latitude, longitude), em um município, em uma data, em um minuto específico}. Cada registro representa uma detecção individual de foco de incêndio, com suas características de risco, potência radiativa e contexto de seca. A granularidade de minuto é essencial para preservar a distribuição temporal fina dos eventos de queimadas, permitindo análises de intensidade e padrões de propagação.

\subsubsection{Fatos}
\begin{itemize}
    \item \textbf{risco\_fogo}: float, não aditivo.
    \begin{itemize}
        \item \textit{Agregação temporal}: MÉDIA
        \item \textit{Agregação espacial}: MÉDIA
        \item \textit{Justificativa}: Representa uma medida de risco que deve ser calculada como média para manter representatividade estatística.
    \end{itemize}
    
    \item \textbf{potencia\_radiativa\_fogo}: float, aditivo.
    \begin{itemize}
        \item \textit{Agregação temporal}: SOMA
        \item \textit{Agregação espacial}: SOMA
        \item \textit{Justificativa}: Representa energia liberada em megawatts, que pode ser somada para obter energia total por período ou região.
    \end{itemize}
    
    \item \textbf{dias\_sem\_chuva}: int, semi-aditivo.
    \begin{itemize}
        \item \textit{Agregação temporal}: MAX (para obter o período mais longo sem chuva)
        \item \textit{Agregação espacial}: MÉDIA (para obter condição média da região)
        \item \textit{Justificativa}: Temporalmente, o valor máximo indica o período crítico de seca; espacialmente, a média fornece condição representativa da região.
    \end{itemize}
\end{itemize}

\subsubsection{Dimensões}
A tabela de fatos se conecta às seguintes dimensões conformadas:
\begin{itemize}
    \item \textbf{dim\_data}
    \item \textbf{dim\_local}
    \item \textbf{dim\_horario} (com granularidade de minuto)
\end{itemize}

\begin{figure}[H]
    \centering
    \includegraphics[width=0.8\textwidth]{diagrams/queimadas_schema.png}
    \caption{Esquema Estrela para Queimadas}
    \label{fig:queimadas_schema}
\end{figure}

\subsection{Fato Clima}

\subsubsection{Granularidade}
A tabela de fatos Clima opera no grão de \textbf{medições climáticas agregadas por município, por data, por hora}. Cada registro representa as condições climáticas médias observadas em um município durante uma hora específica, incluindo temperatura, precipitação, umidade, vento e qualidade do ar. A granularidade horária é apropriada para dados climáticos, pois representa a resolução temporal típica das estações meteorológicas e dos modelos climáticos, equilibrando precisão analítica com volume de dados.

\subsubsection{Fatos}
\begin{itemize}
    \item \textbf{temperatura}: float, não aditivo.
    \begin{itemize}
        \item \textit{Agregação temporal}: MÉDIA
        \item \textit{Agregação espacial}: MÉDIA
        \item \textit{Justificativa}: Temperatura é uma medida intensiva que deve ser calculada como média ponderada.
    \end{itemize}
    
    \item \textbf{precipitacao}: float, semi-aditivo.
    \begin{itemize}
        \item \textit{Agregação temporal}: SOMA (para precipitação acumulada no período)
        \item \textit{Agregação espacial}: MÉDIA (para precipitação média da região)
        \item \textit{Justificativa}: Temporalmente representa acumulação; espacialmente representa condição média regional.
    \end{itemize}
    
    \item \textbf{umidade\_relativa}: float, não aditivo.
    \begin{itemize}
        \item \textit{Agregação temporal}: MÉDIA
        \item \textit{Agregação espacial}: MÉDIA
        \item \textit{Justificativa}: Medida intensiva expressa em percentual, agregada por média.
    \end{itemize}
    
    \item \textbf{vento\_velocidade}: float, não aditivo.
    \begin{itemize}
        \item \textit{Agregação temporal}: MÉDIA
        \item \textit{Agregação espacial}: MÉDIA
        \item \textit{Justificativa}: Velocidade deve ser agregada por média para manter representatividade física.
    \end{itemize}
    
    \item \textbf{vento\_direcao}: int, não aditivo.
    \begin{itemize}
        \item \textit{Agregação temporal}: MODA ou MÉDIA\_CIRCULAR
        \item \textit{Agregação espacial}: MODA ou MÉDIA\_CIRCULAR
        \item \textit{Justificativa}: Direção em graus azimutais requer agregação circular ou moda para manter significado físico.
    \end{itemize}
    
    \item \textbf{co\_ppb}: float, não aditivo.
    \begin{itemize}
        \item \textit{Agregação temporal}: MÉDIA
        \item \textit{Agregação espacial}: MÉDIA
        \item \textit{Justificativa}: Concentração de monóxido de carbono é medida intensiva, agregada por média para representar condições médias de qualidade do ar.
    \end{itemize}
    
    \item \textbf{no2\_ppb}: float, não aditivo.
    \begin{itemize}
        \item \textit{Agregação temporal}: MÉDIA
        \item \textit{Agregação espacial}: MÉDIA
        \item \textit{Justificativa}: Concentração de dióxido de nitrogênio é medida intensiva, agregada por média para análises de qualidade do ar.
    \end{itemize}
    
    \item \textbf{o3\_ppb}: float, não aditivo.
    \begin{itemize}
        \item \textit{Agregação temporal}: MÉDIA
        \item \textit{Agregação espacial}: MÉDIA
        \item \textit{Justificativa}: Concentração de ozônio é medida intensiva, crucial para análises de qualidade do ar e correlações com queimadas.
    \end{itemize}
    
    \item \textbf{pm25\_ugm3}: float, não aditivo.
    \begin{itemize}
        \item \textit{Agregação temporal}: MÉDIA
        \item \textit{Agregação espacial}: MÉDIA
        \item \textit{Justificativa}: Material particulado fino (PM2.5) é indicador crítico de qualidade do ar, especialmente relevante em áreas com queimadas.
    \end{itemize}
    
    \item \textbf{so2\_ugm3}: float, não aditivo.
    \begin{itemize}
        \item \textit{Agregação temporal}: MÉDIA
        \item \textit{Agregação espacial}: MÉDIA
        \item \textit{Justificativa}: Concentração de dióxido de enxofre é medida intensiva importante para análises de qualidade do ar e impactos ambientais.
    \end{itemize}
\end{itemize}

\subsubsection{Dimensões}
A tabela de fatos se conecta às seguintes dimensões conformadas:
\begin{itemize}
    \item \textbf{dim\_data}
    \item \textbf{dim\_local}
    \item \textbf{dim\_horario} (com granularidade de hora, via \textit{bridge\_horario})
\end{itemize}

\begin{figure}[H]
    \centering
    \includegraphics[width=0.8\textwidth]{diagrams/clima_schema.png}
    \caption{Esquema Estrela para Clima}
    \label{fig:clima_schema}
\end{figure}

\subsection{Constelação de Fatos}

A constelação final integra os dois fatos através das dimensões conformadas \textit{dim\_data} e \textit{dim\_local}, e utiliza a tabela ponte \textit{bridge\_horario} para resolver a diferença de granularidade temporal. Esta arquitetura permite análises como:
\begin{itemize}
    \item Correlação entre condições climáticas e ocorrência de queimadas por município e período.
    \item Análise temporal em múltiplas granularidades (minuto, hora, dia, mês, etc.).
    \item Drill-across entre fatos para análises integradas de clima e queimadas.
\end{itemize}

\begin{figure}[H]
    \centering
    \includegraphics[width=\textwidth]{diagrams/full_schema.png}
    \caption{Constelação de Fatos com Dimensões Conformadas e Tabela Ponte}
    \label{fig:full_schema}
\end{figure}


\section{Consultas}

A seguir, são apresentados exemplos de operações OLAP que podem ser executadas sobre a constelação de fatos para extrair insights analíticos. As consultas apresentadas são de alto nível, focando nos conceitos e hierarquias, sendo independentes de implementação específica.

\subsection{Drill-Down com agregação de contagem}
\textit{Visão: Número de Focos de Incêndio por Estado por Ano} \(\rightarrow\) \textit{Visão: Número de Focos de Incêndio por Município por Ano}.
\subsubsection{Motivação} Após identificar Estados com maior incidência de focos de incêndio (contagem de registros na tabela de fatos Queimadas), o analista pode "descer" na hierarquia geográfica de \textit{dim\_local} (estado \(\rightarrow\) município) para avaliar quais municípios são os maiores contribuintes, permitindo o direcionamento de ações de fiscalização e prevenção.

\subsection{Roll-Up com agregação de média}
\textit{Visão: Média de Precipitação por Estado por Mês} \(\rightarrow\) \textit{Visão: Média de Precipitação por Região por Ano}.
\subsubsection{Motivação} Agregar dados mensais para uma visão anual (mês \(\rightarrow\) ano na hierarquia de \textit{dim\_data}) e de estado para região (estado \(\rightarrow\) região na hierarquia de \textit{dim\_local}) permite a identificação de tendências macro, suavizando variações de curto prazo e revelando padrões climáticos de larga escala. A agregação de \textit{precipitacao} utiliza média espacial (conforme definido no esquema Clima).

\subsection{Pivot}
\textit{Visão: Total de Potência Radiativa por Estado por Mês} \(\rightarrow\) \textit{Visão: Total de Potência Radiativa por Mês por Estado}.
\subsubsection{Motivação} A primeira visão facilita a análise da série temporal do impacto das queimadas para cada estado, utilizando a medida aditiva \textit{potencia\_radiativa\_fogo} do fato Queimadas. Ao "pivotar" a tabela (reorientando as dimensões \textit{dim\_local} e \textit{dim\_data}), a nova visão facilita a comparação do impacto entre os estados para um mesmo mês, identificando áreas críticas em cada período.

\subsection{Slice com agregação de média espacial}
\textit{Visão: Precipitação por Estado por Mês} \(\rightarrow\) \textit{Visão: Precipitação Média por Estado para o mês de Setembro}.
\subsubsection{Motivação} Permite isolar um "pedaço" do cubo de dados para análise focada. Neste caso, seleciona-se o mês de Setembro através de \textit{dim\_data} (tipicamente um mês crítico de seca) para analisar a precipitação média espacial (conforme agregação definida no fato Clima) em cada estado, auxiliando no planejamento de ações preventivas. Esta operação utiliza as dimensões conformadas para filtrar o período de interesse.

\subsection{Drill-Down com análise hierárquica por bioma}
\textit{Visão: Média de Risco de Fogo por Estado por Ano} \(\rightarrow\) \textit{Visão: Média de Risco de Fogo por Município por Bioma por Ano}.
\subsubsection{Motivação} Demonstra a utilização da hierarquia geográfica especializada do fato Queimadas: através de \textit{dim\_ponto}, que mantém informações de bioma e se relaciona hierarquicamente com \textit{dim\_local}. Esta consulta permite descer do nível de estado para município, adicionando a dimensão de bioma, revelando que municípios específicos podem apresentar riscos distintos em diferentes ecossistemas (Amazônia, Cerrado, etc.). A medida \textit{risco\_fogo} utiliza agregação por média, conforme definido no esquema. Esta análise é essencial para políticas de conservação direcionadas por tipo de bioma.

\subsection{Slice multi-dimensional com qualidade do ar}
\textit{Visão: Concentrações Médias de PM2.5, CO e O3 por Município por Dia} \(\rightarrow\) \textit{Visão: Concentrações Médias de PM2.5, CO e O3 para Municípios da Região Norte durante a Estação de Seca (Julho-Setembro)}.
\subsubsection{Motivação} Demonstra uma operação de slice em múltiplas dimensões do fato Clima, isolando região geográfica específica através de \textit{dim\_local} e período sazonal através de \textit{dim\_data} (utilizando os atributos analíticos de estação e mês). As medidas de qualidade do ar (\textit{pm25\_ugm3}, \textit{co\_ppb}, \textit{o3\_ppb}) são não-aditivas e agregadas por média. Esta análise é crítica para avaliar impactos ambientais e de saúde pública em períodos de maior incidência de queimadas, permitindo comparações entre diferentes poluentes atmosféricos simultaneamente.

\subsection{Drill-Across com agregação de média}
\textit{Visão: Média de Umidade Relativa por Região por Trimestre} + \textit{Visão: Média de Potência Radiativa por Região por Trimestre} \(\rightarrow\) \textit{Visão: Média de Umidade Relativa e Média de Potência Radiativa por Região por Trimestre}.
\subsubsection{Motivação} Permite a análise de correlação entre métricas de diferentes tabelas de fato (Clima e Queimadas) através das \textbf{dimensões conformadas} \textit{dim\_local} e \textit{dim\_data}. As medidas \textit{umidade\_relativa} (fato Clima) e \textit{potencia\_radiativa\_fogo} (fato Queimadas) podem ser agregadas simultaneamente utilizando a hierarquia região de \textit{dim\_local} e trimestre de \textit{dim\_data}. A \textit{bridge\_horario} permite integrar dados de diferentes granularidades temporais (hora para Clima, minuto para Queimadas) ao agregar para níveis superiores. Esta consulta é fundamental para validar hipóteses, como a de que trimestres com menor umidade relativa apresentam incêndios com maior potência radiativa.

\subsection{Análise de Pico de Queimadas por Hora usando Bridge Table}
\textit{Visão: Contagem de Focos de Incêndio por Município por Hora} com correlação \textit{Temperatura e PM2.5 Médias por Município por Hora}.
\subsubsection{Motivação} Esta consulta demonstra explicitamente o uso da \textbf{tabela ponte} (\textit{bridge\_horario}) para integrar eventos de queimadas (granularidade de minuto) com dados climáticos (granularidade de hora). A tabela ponte realiza o mapeamento \textbf{60:1} entre minutos e horas, permitindo agregar múltiplos focos de incêndio detectados em diferentes minutos da mesma hora. Por exemplo, se um município teve focos detectados às 14:23, 14:37 e 14:54, a \textit{bridge\_horario} permite agregar esses três eventos (via \textit{horario\_minuto\_fk}) para a hora 14:00 (via \textit{horario\_hora\_fk}), possibilitando a contagem total de focos por hora e sua correlação com as condições climáticas médias dessa hora (temperatura e concentração de PM2.5). Esta análise é essencial para identificar padrões temporais de intensificação de queimadas e seus impactos imediatos na qualidade do ar, revelando, por exemplo, que determinadas horas do dia apresentam maior concentração de focos e consequentemente piores níveis de poluição atmosférica.


\newpage 

\bibliography{referencias}

\nocite{sisam_inpe}
\nocite{queimadas_inpe}


\postextual

\end{document}
